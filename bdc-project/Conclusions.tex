\section*{Conclusions}
\phantomsection

For the course project I have researched the control structures in PL/SQL and T-SQL and for the practice part I have used Oracle and Microsoft SQL Server.\newline

First of all, there are three types of control structures:
conditional selection statements, which run different statements for different data values(If and Case), loop statements, which run the same statements with a series of different data values(FOR, While) and sequential control statements, like GOTO, which goes to a specified statement, and NULL, which does nothing.\newline

These have the same logic for either Oracle and MSSSQL and other programming languages as well. But between Oracle and MSSQL there are some differences when it comes to their implementation and it is not only about the syntax. For example, there are some keywords that are used only in MSSQL, like BREAK and CONTINUE whereas in Oracle it is used the EXIT and EXIT WHEN keywords. Also, one main difference is in the iteration part - in MSSQL there is no FOR loop and you can simulate it using a while loop instead. \newline

During this course project, I have learned each structure for both MSSQL and Oracle and then I have deduced the similarities and the differences for them. I have created several example in Oracle and MSSQL in order to show how they are implemented and to see the result when executing them. \newline

I think that control structures are a very powerful tool which allow developers to do much more with a structured and logical source code. That's why it is important to understand the concept of control structures, to know how to use them and when it is necessary to use. When you understand them, it is easier to use either in Oracle or in MSSQL. Using the documentation that they provide, we can easily find the syntax and some specific rules for the context used. \newline

Personally, I have got a lot of knowledge doing this course project and when doing the practical part, it also has helped me in understanding better how to work in Oracle as well as in SQL. This kind of course projects not only that teaches you some information and how to implement it,but Besides this, it teaches how to research an unknown subject, to use the documentation from the official parts and to structure everything in a  report.  


\clearpage